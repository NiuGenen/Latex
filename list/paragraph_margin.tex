\documentclass[UTF8]{ctexart}

\newcommand{\ParMargin}[2]{
\begin{list}{}{
\setlength{\topsep}{0ex}
\setlength{\parsep}{0ex}
\setlength{\leftmargin}{#1em}
\setlength{\itemindent}{0em}
\setlength{\itemsep}{0ex}
}
#2
\end{list}
}

\begin{document}

汪淼驱车沿京密路到密云县,再转至黑龙潭,又走了一段盘山路,便到达中科院国家天文观测
中心的射电天文观测基地。他看到二十八面直径为九米的抛物面天线在暮色中一字排开,像一
排壮观的钢铁植物,2006年建成的两台高大的五十米口径射电望远镜天线矗立在这排九米天线
的尽头,车驶近后,它们令汪淼不由想起了那张杨冬母女合影的背景。\par
\ParMargin{-2}{
    \item 但叶文洁的学生从事的项目与这些射电望远镜没有什么关系,沙瑞山博士的实验室主要接收三
        颗卫星的观测数据:1989年11月升空、即将淘汰的微波背景探测卫星COBE。2003年发射的威尔
        金森微波各向异性探测卫星WMAP和2007年欧洲航天局发射的普朗克高精度宇宙微波背景探测卫
        星Planek。
}
\ParMargin{0}{
    \item 宇宙整体的微波背景辐射频谱非常精确地符合温度为2.726K黑体辐射谱,具有高度
        各向同性,但在不同局部也存在大约百万分之五涨落的幅度。沙瑞山的工作就是根据
        卫星观测数据,重新绘制一幅更精确的全宇宙微波辐射背景图。这个实验室不大,主
        机房中挤满了卫星数据接收设备,有三台终端分别显示来自三颗卫星的数据。\par
}\par
\ParMargin{2}{
    \item 沙瑞山见到汪淼,立刻表现出了那种长期在寂寞之地工作的人见到来客的热情,问
        他想了解哪方面的观测数据。\par
}\par
\ParMargin{4}{
    \item “我想观测宇宙背景辐射的整体波动。”
    \item “您能……说具体些吗?”沙瑞山看汪淼的眼神变得奇怪起来。
    \item “就是,宇宙3K微波背景辐射整体上的各向同性的波动,振幅在百分之一至百分之
        五之间。”
}\par
\ParMargin{6}{
    \item 沙瑞山笑笑,早在本世纪初,密云射电天文基地就对游客开放参观,为挣些外快,
        沙瑞山时常做些导游或讲座的事,这种笑容就是他回答游客(他已适应了他们那骇
        人的科盲)问题时常常露出的。“汪先生,您……不是搞这个专业的吧?”
}\par

\end{document}
